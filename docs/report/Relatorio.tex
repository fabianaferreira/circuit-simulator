\documentclass[a4paper, 12pt]{article}
\usepackage[portuguese]{babel}
\usepackage[utf8]{inputenc}
\usepackage{amsmath}
\usepackage{indentfirst}
\usepackage{graphicx}
\usepackage[colorinlistoftodos]{todonotes}

\begin{document}
\begin{titlepage}
    \begin{center}
        \huge{Universidade Federal do Rio de Janeiro}

\vspace{10pt}
\begin{figure}[!ht]
\centering
\includegraphics[width=3cm]{pictures/minerva.png}
\hspace{3cm}
\includegraphics[height=3cm, width=7cm]{pictures/poli.jpg}
\end{figure}
        \vspace{85pt}
        \textbf{\LARGE{Trabalho de Circuitos Eletricos II}}
        \vspace{160pt}
    \end{center}
    
    \begin{flushleft}
        \begin{tabbing}
            Alunos\qquad\qquad\= Fabiana Ferreira Fonseca\\
            \> Leonardo Barreto Alvez\\
            \> Vinícius dos Santos Mello\\
            Professor\> Antônio Carlos Moreirão de Queiroz \\
            Horário\> Qua - 12:00-15:00\\
            		\> Sex - 13:00-15:00
        
    \end{tabbing}
          
    \end{flushleft}
    
    \begin{center}
        \vspace{\fill}
        Rio de Janeiro, 20 de Novembro de 2017
    \end{center}
\end{titlepage}
%%%%%%%%%%%%%%%%%%%%%%%%%%%%%%%%%%%%%%%%%%%%%%%%%%%%%%%%%%%

\newpage
\tableofcontents
\thispagestyle{empty}

\newpage
\pagenumbering{arabic}

%%%%%%%%%%%%%%%%%%%%%%%%%%%%%%%%%%%%%%%%%%%%%%%%%%%%%%%%%
%%%%%%%%%%%%%%%%%%%%%%%%%%%%%%%%%%%%%%%%%%%%%%%%%%%%
\section{Objetivo}

O objetivo deste experimento é a realização 
\section{Equipamentos Utilizados}

Para o experimento em questão foram utilizados dois geradores síncronos em que um atuava como motor para configurar um conjunto motor síncrono-gerador síncrono. A função do motor síncrono é a de fornecer potência mecânica ao eixo do gerador para que, então, seja possível realizar os ensaios no gerador síncrono. As placas de dados do motor e do gerador síncrono estão ilustrados nas Figuras 

\section{Procedimento Experimental}

Descrição procedimento

\subsection{subseção1}

tabela exemplo

\begin{table}[htb]
\centering
\begin{tabular}{c|c}
$I_{exc}$ [A] & $V_{a}$ [V] \\ \hline
0.188   & 2.63      \\\hline
0.88 & 20.55    \\\hline
1.83 & 39.26 \\\hline
2.66 & 55.19 \\\hline
3.78 & 74.94 \\\hline
5.00 & 93.70 \\\hline
6.28 & 109.87 \\\hline
7.55 & 122.28 \\\hline
10.05 & 139.03 \\\hline
11.98 & 148.55 \\\hline
13.30 & 153.92 
\end{tabular}
\caption{Pontos coletados para o ensaio à vazio;}
\label{vazio}
\end{table}



\section{Análise dos Resultados}


aálise dados
\section{Conclusão}

conclusao



\newpage
\section{Referências}

[1] Chapman, S.J. -- Electric Machinery Fundamentals, 4th Edition;

[2] Fitzgerald, A. E. -- Máquinas Elétricas, 2da Edição;

\end{document}